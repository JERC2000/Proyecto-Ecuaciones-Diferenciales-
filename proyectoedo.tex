\documentclass[12pt]{report}
\usepackage[spanish]{babel}

\begin{document}

\begin{titlepage}
\centering
{\bfseries\LARGE UNIVERSIDAD DEL CAUCA \par}
\vspace{1cm}
{\scshape\Huge Ecuaciones de Bessel y Legendre\par}
\vspace{3cm}
{\itshape\Large Proyecto Ecuaciones Diferenciales \par}
\vfill
{\Large Presentado por :\par}
{\large Juan Esteban Rivera Cortes \par}
\vfill
{\large Agosto 2022 \par}
\end{titlepage}
\tableofcontents

\chapter{Ecuaciones de Bessel}

Una Ecuaci\'on de Bessel de orden v tiene la forma:
\begin{equation}
x^{2}y^{''}+xy{'}+(x^{2}-v^{2})=0
\end{equation}
Para hallar las soluciones de la ecuaci\'on $(1.1)$, haremos desarrollo en series de potencias.\\
Como el punto x=0 es un punto singular regular de la ecuacion de Bessel, indica que admite solucion en serie de potencias de la forma:\\
\vspace{0.3 cm}
$y=\sum_{n=0}^{\infty}c_{n}x^{n+r},$derivando: $y^{'}=\sum_{n=0}^{\infty}c_{n}(n+r)x^{n+r-1}$,\\
\vspace{0.3 cm}
$y^{''}=\sum_{n=0}^{\infty}c_{n}(n+r)(n+r+1)x^{n+r-2}$ \\
\vspace{0.3 cm}
al sustituir en $(1.1)$\\
\vspace{0.3 cm}
$x^{2}y^{''}+xy{'}+(x^{2}-v^{2})=x^{2}\sum_{n=2}^{\infty}c_{n}(n+r)(n+r+1)x^{n+r-2}+$\\$x\sum_{n=1}^{\infty}c_{n}(n+r)x^{n+r-1}+(x^{2}-v^{2})\sum_{n=0}^{\infty}c_{n}x^{n+r}$\\
\vspace{0.3 cm}
$=\sum_{n=0}^{\infty}c_{n}(n+r)(n+r-1)x^{n+r}+\sum_{n=0}^{\infty}c_{n}(n+r)x^{n+r}+\sum_{n=0}^{\infty}c_{n}x^{n+r+2}-v^{2}\sum_{n=0}^{\infty}c_{n}x^{n+r}$\\
\vspace{0.3 cm}
$=c_{0}(r^{2}-r+r-v^{2})x^{r}+x^{r}\sum_{n=1}^{\infty}c_{n}[(n+r)(n+r-1)+(n+r)-v^{2}]x^{n}+x^{r}\sum_{n=0}^{\infty}c_{n}x^{n+2}$\\
\vspace{0.3 cm}
$=c_{0}(r^{2}-v^{2})x^{r}+x^{r}\sum_{n=1}^{\infty}c_{n}[(n+r)^{2}-v^{2}]x^{n}+x^{r}\sum_{n=0}^{\infty}c_{n}x^{n+2}$\\
\vspace{0.3 cm}
por tanto la ecuacion indicial es: $r^{2}-v^{2}=0$ y las raices indiciales son $r_{1}=v$ y $r_{2}=-v$, si r toma el valor v,
al resolver el cuadrado,simplificar y factorizar, se obtiene:\\
\begin{center}
$x^{v}\sum_{n=1}^{\infty}c_{n}n(n+2v)x^{n}+x^{v}\sum_{n=0}^{\infty}c_{n}x^{n+2}$\vspace{0,5cm}
$x^{v}[(1+2v)c_{1}x+\sum_{n=2}^{\infty}c_{n}n(n+2v)x^{n}+\sum_{n=0}^{\infty}c_{n}x^{n+2}]$
\end{center}
haciendo las siguientes sustituciones $k=n-2$ y $k=n$ respectivamente, se obtiene:
\begin{center}
$x^{v}[(1+2v)c_{1}x+\sum_{k=0}^{\infty}[c_{k+2}(k+2)(k+2+2v)+c_{k}]x^{k+2}]$
\end{center}
dado que $(1+2v)c_{1}x=0$ se obtiene que $c_{1}=0,$ al despejar $c_{k+2}$, se tiene:
\begin{center}
$c_{k+2}=\frac{-c_{k}}{(k+2)(k+2+2v)}, \,para\,k=0,1,2,...$
\end{center}
realizando las iteraciones:\\
\begin{center}
$c_{2}=\frac{-c_{0}}{(2)(2+2v)}=\frac{-c_{0}}{2^{2}(1+v)}$\\
\vspace{0.5cm}
$c_{3}=\frac{-c_{1}}{(3)(3+2v)}=0$
\vspace{0.5cm}\\
$c_{4}=\frac{-c_{2}}{(4)(4+2v)}=\frac{c_{0}}{2^{4}1\cdot 2(1+v)(2+v)}$
\vspace{0.5cm}\\
$c_{5}=\frac{-c_{3}}{(5)(5+2v)}=0$
\vspace{0.5cm}\\
$c_{6}=\frac{-c_{4}}{(6)(6+2v)}=\frac{-c_{0}}{2^{6}1\cdot 2\cdot 3(1+v)(2+v)(3+v)}$
\vspace{0.5cm}\\
\end{center}
en general, se tienen solo los coheficientes pares:\\
\begin{equation}
c_{2k}=\frac{(-1)^{k}c_{0}}{2^{2k}k!(1+v)(2+v)\cdots (k+v)}
\end{equation}\\
tomando:
\begin{equation}
c_{0}=\frac{1}{2^{v}\Gamma (1+v)}
\end{equation}
al sustituir $(1.3)$ en $(1.2)$ y hacer $n=k$
\begin{equation}
c_{2n}=\frac{(-1)^{n}}{2^{2n+v}n!(1+v)(2+v)\cdots (n+v)\Gamma(1+v)}
\end{equation}
se sabe que la funcion Gamma tiene la propiedad:\\
\begin{equation}
x\Gamma(x)=\Gamma(x+1)
\end{equation}
\vspace{0.3 cm}
 de esta manera aplicando $(1.5)$ de forma iterativa:\\
 \vspace{0.3 cm}
 $(n+v)\cdots(2+v)(1+v)\Gamma(1+v)$\\
 \vspace{0.3 cm}
 $=(n+v)\cdots(2+v)\Gamma(2+v+1)$\\
 \vspace{0.3 cm}
 $=(n+v)\cdots(3+v)\Gamma(3+v+1)$\\
 \vspace{0.3 cm}
 en general se tiene:\\
 \vspace{0.3 cm}
 $=\Gamma(n+v+1)$\\
 \vspace{0.3 cm}
 sustituyendo en $(1.4)$, se obtiene:\\
 \vspace{0.3 cm}
$ c_{2n}=\frac{(-1)^{n}}{2^{2n+v}n!\Gamma(n+v+1)}$\\
a la funci\'on que se forma por $y=\sum_{n=0}^{\infty}c_{2n}x^{2n+v}$, la llamaremos:\\
\begin{center}
$J_{v}(x)=\sum_{n=0}^{\infty}\frac{(-1)^{n}}{n!\Gamma(n+v+1)}(\frac{x}{2})^{2n+v}$
\end{center}
 si $v>0$ y converge en el intervalo $(0,\infty)$\\
 de aqu\'i podemos hallar la solucion correspondiente a $r_{2}=-v$\\
\begin{center}
$J_{-v}(x)=\sum_{n=0}^{\infty}\frac{(-1)^{n}}{n!\Gamma(n-v+1)}(\frac{x}{2})^{2n-v}$\\
 
\end{center}
para plantear la solucion general se deben atender dos casos:\\
CASO I.\\
$r_{1}-r_{2}=2v$ un n\'umero no entero\\
aqu\'i la soluci\'on general seria $y=c_{1}J_{v}(x)+c_{2}J_{-v}(x)$\\
CASO II\\
$r_{1}-r_{2}=2v$ un n\'umero entero\\
para ello se podrian presentar dos posibilidades\\
que v=m sea un entero,en este caso podria existir una solucion de la forma:\\
\begin{center}
$y_{1}(x)=\sum_{n=0}^{\infty}c_{n}x^{n+r_{1}}$  con   $c_{0}\neq 0$\\
$y_{2}=Cy_{1}lnx+\sum_{n=0}^{\infty}b_{n}x^{n+r_{2}}$ con $b_{0}\neq 0$
\end{center}
tambien se puede dar el caso donde v es la mitad de un entero impar, cuyo caso no tendria ningun problema ya que $J_{v}$ y $J_{-v}$ son linalmente independientes.\\
por tanto la soluci\'on de $(1.1)$ es\\
\begin{equation}
y=c_{1}J_{v}(x)+c_{2}J_{-v}(x)\,si\,\,v\,\,no\,\,es\,\,un\,\,entero
\end{equation}\\
\section{Apliaciones}
\subsection{Ecuaciones Diferenciales}
Permite transformar algunas ecuaciones diferenciales de orden 2, para poder estudiar problemas en la frontera
\subsection{Musica}
Sirven para representar la defleccion de un instrumento de precusion que es sometido a una vibracion
\subsection{Ingenieria Civil}
Sirven para representar la vibracion de estructuras con amortiguacion sometidas a fuerzas externas 
\subsection{Fisica}
Sirven para represtar la solucion de la ecuacion de schrodinger,campos electricos y mecanica cuantica
\chapter{Ecuaciones de Legendre}
la ecuaci\'on de legendre de orden n tiene la forma:\\
\begin{equation}
(1-x^{2})y^{''}-2xy^{'}+n(n+1)y=0
\end{equation}
para la solucion de esta ecuaci\'on, primero analicemos que  el punto $x=0$ es un punto ordinario, por tanto admite solucion en series de potencias de la forma:\\
\vspace{0.3cm}
$y=\sum_{k=o}^{\infty}c_{k}x^{k}$, derivando\\
\vspace{0.3cm}
$y^{'}=\sum_{k=1}^{\infty}c_{k}kx^{k-1}$\\
\vspace{0.3cm}
$y^{''}=\sum_{k=2}^{\infty}c_{k}k(k-1)x^{k-2}$\\
\vspace{0.3cm}
sustituyendo en la ecuci\'on $(2.1)$ se tiene:\\
\vspace{0.3cm}
$(1-x^{2})\sum_{k=2}^{\infty}c_{k}k(k-1)x^{k-2}-2x\sum_{k=1}^{\infty}c_{k}kx^{k-1}+n(n+1)\sum_{k=o}^{\infty}c_{k}x^{k}=0$\\
\vspace{0.3cm}
$=\sum_{k=2}^{\infty}c_{k}k(k-1)x^{k-2}-x^{2}\sum_{k=2}^{\infty}c_{k}k(k-1)x^{k-2}-2x\sum_{k=1}^{\infty}c_{k}kx^{k-1}$\\
\vspace{0.3cm}
$+n(n+1)\sum_{k=o}^{\infty}c_{k}x^{k}$\\
\vspace{0.3cm}
$=\sum_{k=2}^{\infty}c_{k}k(k-1)x^{k-2}-\sum_{k=2}^{\infty}c_{k}k(k-1)x^{k}-2\sum_{k=1}^{\infty}c_{k}kx^{k}$\\
\vspace{0.3cm}
$+n(n+1)\sum_{k=o}^{\infty}c_{k}x^{k}$\\
\vspace{0.3cm}
al igualar las potencias de x y los indices de la sumatoria obtenemos:\\
\vspace{0.3cm}
$=[n(n+1)c_{0}+2c_{2}]+[(n-1)(n+2)c_{1}+6c_{3}]x+$\\
\vspace{0.3cm}
$\sum_{k=2}^{\infty}[(k+2)(k+1)c_{k+2}+(n-k)(n+k+1)c_{k}]x^{k}=0$\\
\vspace{0.3cm}
como se busca que los coheficientes sean cero,se tiene:\\
\vspace{0.3cm}
$n(n+1)c_{0}+2c_{2}=0\rightarrow c_{2}=\frac{-n(n+1)c_{0}}{2!}$\\
\vspace{0.3cm}
$(n-1)(n+2)c_{1}+6c_{3}=0\rightarrow c_{3}=\frac{-(n-1)(n+2)c_{1}}{3!}$\\
\vspace{0.3cm}
$(k+2)(k+1)c_{k+2}+(n-k)(n+k+1)c_{k}\rightarrow c_{k+2}=\frac{-(n-k)(n+k+1)c_{k}}{(k+2)(k+1)}$ para $k=2,3,4,...$\\
\vspace{0.3cm}
aplicando la relaci\'on de recurrencia:\\
\vspace{0.3cm}
$c_{4}=-\frac{(n-2)(n+3)}{4\cdot3}c_{2}=\frac{(n-2)n(n+1)(n+3)}{4!}c_{0}$\\
\vspace{0.3cm}
$c_{5}=-\frac{(n-3)(n+4)}{5\cdot4}c_{3}=\frac{(n-3)(n-1)(n+2)(n+4)}{5!}c_{1}$\\
\vspace{0.3cm}
$c_{6}=-\frac{(n-4)(n+5)}{6\cdot5}c_{4}=-\frac{(n-4)(n-2)n(n+1)(n+3)(n+5)}{6!}c_{0}$\\
\vspace{0.3cm}
$c_{7}=-\frac{(n-5)(n+6)}{7\cdot6}c_{5}=-\frac{(n-5)(n-3)(n-1)(n+2)(n+4)(n+6)}{7!}c_{1}$\\
la serie de potencias converge para $|x|<1$\\
\vspace{0.3cm}
de este modo, las dos soluciones linealmente independientes son:\\
\vspace{0.3cm}
$y_{1}(x)=c_{0}[1-\frac{n(n+1)}{2!}x^{2}+\frac{(n-2)n(n+1)(n+3)}{4!}x^{4}-\frac{(n-4)(n-2)n(n+1)(n+3)(n+5)}{6!}x^{6}+\cdots]$\\
\vspace{0.3cm}
$y_{2}(x)=c_{1}[x-\frac{(n-1)(n+2)}{3!}x^{3}+\frac{(n-3)(n-1)(n+2)(n+4)}{5!}x^{5}-\frac{(n-5)(n-3)(n-1)(n+2)(n+4)(n+6)}{7!}x^{7}\cdots]$\\
hay que notar en ambas soluciones que dependiendo del valor de n, una de las dos soluciones sera un polinomio con una cantidad finita de t\'erminos\\.
\section{Aplicaciones}
\subsection{Fisica}
Permite expresar la factorizacion del operador hamiltoniano y generalizando estos operadores usados en la fisica cuantica
\subsection{Matematicas}
Permite representar Armonicos esfericos 
\begin{thebibliography}{0}
\bibitem{Denis G. zill 2009}Denis G. Zill.ecuaciones Diferenciales con aplicaciones de modelado,2009
\bibitem{Lorena Terrios}Lorena terrios.Notas de Ecuaciones Diferenciales,Universidad del Cauca.
\bibitem{internet}https://http://mauricioanderson.com/curso-latex-introduccion-instalacion-estructura/
\end{thebibliography}
\end{document}
